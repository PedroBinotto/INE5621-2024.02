\documentclass[12pt]{article}
\usepackage[portuguese, english]{babel}
\usepackage[T1]{fontenc}
\usepackage[a4paper, left=3cm, right=2cm, top=3cm, bottom=2cm]{geometry}
\usepackage{authblk}
\usepackage{cite}
\usepackage{fontspec}
\usepackage[toc]{glossaries}
\usepackage{graphicx}
\usepackage{hyperref}
\usepackage{setspace}
\usepackage{subcaption}
\usepackage{verbatimbox}
\usepackage[dvipsnames]{xcolor}
\newcommand\myshade{85}
\newcommand{\pprime}{\ensuremath{^{\prime}}}
\addto\captionsportuguese{\renewcommand*\contentsname{Sumário}}

\hypersetup{
    pdftitle={Trabalho Remoto e e-Working - Uma Análise dos Efeitos na Rotina Pessoal, Saúde Mental e Bem-Estar do Trabalhador},
    pdfauthor={Pedro Santi Binotto},
    pdfsubject={Este artigo busca explorar os impactos psicológicos, comportamentais e sociais da adoção do trabalho
    remoto em larga escala e seus efeitos a longo prazo.},
    linkcolor  = black,
    citecolor  = black,
    urlcolor   = black,
    colorlinks = true,
    filecolor=black,        % color of file links
    linktoc=page            % only page is linked
}%

\newglossaryentry{ict} {
  name={ICT},
  description={
    Tecnologia da Informação e Comunicação (\textbf{ICT}, do inglês: \textit{Information and Communications Technology},
    ou \textbf{TIC}, em português) é um termo extensional para tecnologia da informação (\textbf{TI}) que enfatiza o papel das comunicações
    unificadas e a integração de telecomunicações (linhas telefônicas e sinais sem fio) e computadores, bem como software
    empresarial necessário, middleware, armazenamento e audiovisual, que permitem aos usuários acessar, armazenar, transmitir,
    entender e manipular informações \cite{techTarget2023}.
  }
}

\newglossaryentry{teletrabalho} {
  name={teletrabalho},
  description={
    Modalidades de trabalho caracterizadas pela flexibilização do local de trabalho de forma a permitir que o empregado
    realize suas atividades de uma locação remota, externa aos escritórios e/ou complexos de produção corporativos
    \cite{ilo2020,diMartino1990}.
  }
}

\addto\captionsportuguese{\renewcommand*\contentsname{Sumário}}
\onehalfspacing

\selectlanguage{portuguese} 
\title{%
   Trabalho Remoto e e-Working\\
   \large Uma Análise dos Efeitos na Rotina Pessoal, Saúde Mental e Bem-Estar do Trabalhador }
\author[1]{Pedro Santi Binotto [20200634]\thanks{\texttt{pedro.binotto@grad.ufsc.br}}}
\date{\today}
\affil[1]{Departamento de Informática e Estatística, Universidade Federal de Santa Catarina}

\makeglossaries

\begin{document}
\selectlanguage{portuguese} 
\begin{titlepage}
\maketitle
\thispagestyle{empty}

\begin{abstract} \noindent
  O trabalho das modalidades remota e híbrida tem visto crescimento considerável nos últimos anos, 
  especialmente acelerado em consequência das medidas tomadas por empregadores e instituições governamentais em resposta
  à pandemia de COVID-19 no início desta década. Segundo análises de mercado recentes, para setores da economia do
  conhecimento como TI e finanças, vagas completamente remotas representam a maioria absoluta
  do mercado de trabalho em países desenvolvidos (\textit{McKinsey \& Co.}, 2022). Em vista da mudança radical na % TODO : try to inline citation 
  relação das pessoas com o trabalho causada por essa nova dinâmica, este artigo busca explorar os impactos psicológicos,
  comportamentais e sociais da adoção do trabalho remoto em larga escala e seus efeitos a longo prazo.
\end{abstract}

\end{titlepage}

\newpage
\tableofcontents

\newpage
\printglossary[title=Glossário, toctitle=Glossário]

\newpage
\section{Introdução}

Diferentes modalidades de \gls{teletrabalho} têm sido realidade desde a viabilização e popularização das \gls{ict}s
\cite{diMartino1990} no decorrer dos últimos trinta anos \cite{jit2012}, e desde a pandemia de COVID-19, tornou-se comum
para uma grande parcela da força de trabalho de diversas regiões do planeta \cite{apolloTechnical2024}. Este fenômeno tem sido
extensivamente analisado sob perspectivas do empregador \cite{hofschulteBeck2022,sustainability2022} assim como do
empregado \cite{ijerph2021,ntwe2017} de maneira que o resultado destes estudos apresenta uma percepção variada dos
efeitos positivos assim como negativos por ambas as partes.

É no contexto destes estudos que este artigo propõe uma visão geral sintetizada dos efeitos psicológicos cotidianos percebidos
principalmente do referencial do empregado remoto ou híbrido, focando na sustentabilidade e viabilidade dos hábitos
desenvolvidos e analisando os impactos identificados nos padrões de comportamento observados no dia-a-dia do
trabalhador.

\section{Desenvolvimento}
\subsection{Teoria e evidências}

O embasamento teórico para este artigo foi compilado de múltiplos estudos tratando dos sustentabilidade e bem-estar no 
contexto de trabalho remoto conduzidos nos últimos dez anos \cite{mckinsey2022,ijerph2021,ntwe2017,forbes2023},
levando em conta análises da situação pré pandêmica \cite{ntwe2017} assim como estudos conduzidos durante e após a
adoção em massa do \gls{teletrabalho} durante a pandemia\cite{mckinsey2022,ijerph2021,sustainability2022}. 

\subsection{Adoção do trabalho remoto}
\subsubsection{Pré-pandemia}

Os resultados iniciais dos estudos de ambos os períodos indicam as mesmas tendências à curto prazo: as facilidades associadas à
``não-necessidade'' do deslocamento e a flexibilização do espaço de trabalho são refletidas claramente de maneira positiva,
sendo observável em índices elevados de produtividade e compromisso com o trabalho \cite{ntwe2017} por parte dos
empregados. É importante notar que, de modo geral, o trabalho remoto foi historicamente tratado como um privilégio ou ``luxo'' de
determinado empregado antes do \textit{boom} mundial da pandemia de coronavírus em março de 2020 
\cite{ntwe2017,sustainability2022,diMartino1990}. Inversamente, estudos conduzidos após este período refletem a mudança
de contexto trazida pela pandemia e reconhecem a normalização desta modalidade de trabalho
\cite{hofschulteBeck2022,sustainability2022}.

\subsubsection{Adoção do trabalho remoto durante a pandemia de COVID-19}

No decorrer da pandemia, observamos o trabalho remoto na forma de \textit{home office} tornar-se um ``novo
normal'' compulsório para a uma parcela considerável da população economicamente ativa por um período mais longo do que
anteriormente observado \cite{mckinsey2022,sustainability2022,phillips2020}. Assim, ficaram claros alguns dos efeitos adversos da forma com
que o trabalho remoto tem sido conduzido desde então: em muitos casos, o trabalho exclusivamente remoto durante
períodos estendidos está correlacionado à níveis de estresse elevados, redução do equilíbrio entre vida pessoal e profissional,
assim como redução da percepção de satisfação profissional \cite{ijerph2021,forbes2023}.

Apesar disso, nota-se que o trabalho remoto mantém a correlação com níveis mais altos de engajamento e rendimento profissional
\cite{ijerph2021,ntwe2017}. Não obstante, empregadores ainda reportam desafios ao se adaptar à nova realidade dessa espécie
de trabalho, buscando adotar novas medidas de monitoramento motivadas pela percepção de uma linha de comunicação enfraquecida
e/ou falta de observabilidade sobre os empregados \cite{hofschulteBeck2022}. 

\subsubsection{Impacto psicológico do trabalho remoto por períodos estendidos}

Diante desta situação, muitos profissionais remotos reportam a ocorrência de episódios de \textit{burnout} profissional
associado ao constante fluxo de informação através de ferramentas de comunicação (\gls{ict}s), alegando a sensação
de microgerenciamento por parte de seu empregador \cite{forbes2023}.

No campo de trabalho de TI, particularmente na área de desenvolvimento de software, existem estudos mapeando elementos da cultura
corporativa que contribuem para um ambiente de alta tensão relacionados ao microgerenciamento de funcionários e a sensação
de uma falta de segurança profissional \cite{trinkenreichorganization}; tais fatores são apenas exacerbados em cenários
de trabalho remoto que apresentem uma herança cultural destes ambientes \cite{ijerph2021}.

Recursos a respeito dos efeitos provenientes de atividades exclusivamente remotas são abundantes; sensação de
isolamento, dificuldade de acesso à recursos de conhecimento e a percepção de estagnação em relação à progressão de
carreira ou falta de reconhecimento profissional foram alguns dos relatos mais comuns de empregados remotos durante o
período exclusivamente remoto da pandemia \cite{ijerph2021,forbes2023,hofschulteBeck2022}. 

Empregados que possuem cargos de maior responsabilidade também frequentemente reportam dificuldade de se desligar do trabalho
após o expediente, ou que sentem pressão para se fazerem disponíveis em horários fora de sua jornada de trabalho \cite{ijerph2021}.
A combinação destes fatores contribui para o agravamento da sensação de instabilidade, ansiedade, ou falta de segurança
profissional que pode ser observada nos resultados de estudos referentes ao período de trabalho remoto compulsório na pandemia
de COVID-19 \cite{ijerph2021}.

\section{Conclusão}

Após análise dos dados provenientes dos estudos conduzidos, podemos concluir que a possibilidade de realizar as tarefas
remotamente é um passo em direção de um maior nível de liberdade e dignidade para o trabalhador; no entanto, o
movimento de universalização do trabalho na modalidade exclusivamente remota e a falta de um
ambiente dedicado ao trabalho podem ser um agravante para uma relação abusiva para com o trabalho. A oferta de um ambiente
de trabalho digno assim como os meios para o deslocamento do funcionário até sua estação de serviço compõem a estrutura
de suporte ao profissional tanto quanto os demais direitos trabalhistas e devem ser defendidos como tais. 

O período da pandemia provou que o trabalho remoto é não apenas possível, como viável e, para muitos, preferível ou
mesmo necessário; oferece, também, benefícios observáveis da perspectiva do empregador \cite{sustainability2022}.
Por estes motivos, deve ser ofertado sempre que possível para os funcionários e empregadores que podem beneficiar-se
dessa modalidade. 

A adoção compulsória de atividades exclusivamente remotas, no entanto, está relacionada à agravação de ansiedades
relacionadas ao trabalho e à precarização de condições serviço, assim como estagnação e a sensação de
isolamento profissional \cite{ijerph2021,forbes2023}.

A causa dos sentimentos de ansiedade pode estar associada não apenas à natureza do trabalho à distância, mas principalmente
ao modelo de avaliação e supervisão profissional atualmente observado em ambientes de trabalho principalmente nas áreas
de atuação profissional psicológica e/ou técnica \cite{trinkenreichorganization}; isto, juntamente ao isolamento profissional
por períodos estendidos, tem sido associado à deterioração dos níveis de bem-estar psicológico dos empregados
\cite{tulili2023burnout}, evidenciando a necessidade da formalização e aplicação de  padrões de regimento profissional
que garantam que a adoção desta nova dinâmica de trabalho ocorra forma ética, levando em consideração a saúde
psicológica do trabalhador e os possíveis efeitos psicossomáticos desencadeados pela implementação incorreta do
\gls{teletrabalho}.

\newpage
\section{Referências Bibliográficas}
\bibliographystyle{apalike}
\bibliography{references}

\end{document}


