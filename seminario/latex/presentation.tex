\documentclass[t]{beamer}
\usepackage[portuguese]{babel}
\usepackage{biblatex}
\addbibresource{references.bib}

\usetheme{metropolis}           % Use metropolis theme

\setbeamerfont{bibliography item}{size=\tiny}
\setbeamerfont{bibliography entry author}{size=\tiny}
\setbeamerfont{bibliography entry title}{size=\tiny}
\setbeamerfont{bibliography entry location}{size=\tiny}
\setbeamerfont{bibliography entry note}{size=\tiny}

\renewcommand{\footfullcite}[2][]{%
    \footnote{\citeauthor{#2} (\citeyear{#2}): \emph{\citetitle{#2}}.}
}

\title{Consequências da dependência tecnológica do indivíduo moderno}
\author[Short Name (U ABC)]{%
    \texorpdfstring{%
        \begin{columns}
            \column{.3\linewidth}
            \centering
            Pedro Santi Binotto \\ {[}20200634{]}
        \end{columns}
    }
    { Pedro Santi Binotto }
}
\institute{Departamento de Informática e Estatística, Universidade Federal de Santa Catarina}
\date{\today}

\begin{document}
    \frame{\titlepage}
 
    \frame{
      \begin{block}{Definição}

        \begin{description}
          \item[Dependência Tecnológica:] A relação de um indivíduo ou comunidade com algum recurso tecnológico ou
            provedor deste recurso;
        \end{description}

        \begin{itemize}
          \item Relação de natureza assimétrica \footfullcite{secretariat1977};
          \item Caracterizada por \textbf{dominância} ou \textbf{dependência} \footfullcite{kaczynski1995};
        \end{itemize}

      \end{block}
    }
 
    \frame{
      \begin{block}{O que é \textit{Aceleracionismo}?}
        \begin{description}
          \item[Aceleracionismo:] Ideologia que prega que tanto o sistema prevalecente (capitalista), quanto certos 
            processos sociais e tecnológicos historicamente caracterizados, devem ser ampliados, reaproveitados, ou 
            acelerados, a fim de gerar uma mudança social significativa \footfullcite{henkin2016accelerationism};
        \end{description}
      \end{block}
    }
 
    \frame{
      \begin{block}{Tecnologia como um fim}
        \begin{description}
          \item[Tecnocentrismo:] Sistema de valores que se centra na noção do avanço tecnológico como objeto da 
            existência humana \footfullcite{papert1988critique};
        \end{description}
        \vspace{1.0cm}
        \begin{quote}
          ``\textit{Our lives are too fast, we are subject to the accelerating demand that we innovate more, work more, enjoy more,
          produce more, and consume more {[}\dots{]} Hartmut Rosa declares that today we face a ‘totalitarian’ form of social 
          acceleration} \footfullcite{noys2014malign}''.
        \end{quote}
      \end{block}
    }

    \frame{
      \begin{block}{Dependência tecnológica nas escalas macro e individual}
        \begin{itemize}
          \item A dependência tecnológica apresenta-se não apenas nas relações individuais, mas também em escala
            \textit{macro}, manifestando-se na relação assimétrica entre países industrializados e nações em
            desenvolvimento \footfullcite{secretariat1977};
          \item Na escala individual, a dependência tecnológica tipicamente ocorre entre o sujeito (o indivíduo ou uma
            pequena comunidade, por exemplo) e um agente provedor dos recursos, como uma corporação ou governo;
        \end{itemize}
      \end{block}
    }

    \frame{
      \begin{block}{Banalização do senso de impotência}
        \begin{itemize}
          \item Desta forma, a relação de dependência assimétrica causada pelo avanço tecnológico desabalado
            contribui para não apenas a perda de autonomia do indivíduo, mas também promove a percepção de
            impotência perante ao sistema \footfullcite{kaczynski1995};
        \end{itemize}
      \end{block}
    }

    \frame{
      \begin{block}{Alienação do indivíduo}
        \begin{itemize}
          \item Os agentes detentores dos recursos são muitas vezes instituições abstratas poderosas demais para que 
            qualquer pessoa tenha alguma influência sobre este fenômeno \footfullcite{noys2014malign};
        \end{itemize}
      \end{block}
    }

    \frame{
      \begin{block}{Tecnologia e depressão}
        \begin{itemize}
          \item A alienação do indivíduo, combinada com os demais problemas sociais observados em sociedades
            modernas, podem explicar a correlação entre vida urbana em nações industrializadas e um índice acentuado de
            incidência de depressão e outros transtornos psicológicos \footfullcite{XU2023299};
        \end{itemize}
      \end{block}
    }

    \frame{
      \begin{block}{Senso de inadequação}
        \begin{itemize}
          \item Na era digital, em especial, as estratégias de propaganda empregadas e o ritmo acelerado de mudança causada em
            todos os setores da sociedade pela revolução digital também contribui para o senso de inadequação que muitos
            indivíduos reportam vivendo em nações industrialmente desenvolvidas \footfullcite{affiziedoes} 
            \footfullcite{jameshulbert1968};
        \end{itemize}
      \end{block}
    }

    \frame{
      \begin{block}{Acesso à tecnologia}
        \begin{itemize}
          \item Finalmente, a dependência tecnológica é especialmente comprometedora para a parcela da população que
            apresenta dificuldades para acessar os recursos tecnológicos de qual depende;
          \item Estudos recentes indicam que a crescente população idosa apresenta dificuldades para acessar e utilizar
            os recursos digitais sobre os quais dependem para realizar tarefas cotidianas \footfullcite{gitlow2014usage};
        \end{itemize}
      \end{block}
    }

    \frame{
      \begin{block}{Considerações Finais}
        \begin{itemize}
          \item Vivemos em um mundo onde um certo grau de \textbf{dependência tecnológica} é, para a maioria, \textbf{inevitável};
          \item Para aquilo que não é absolutamente necessário utilizar de um recurso que nos torne dependente, cabe a
            cada indivíduo o discernimento de quando renunciar sua autonomia;
          \item O \textbf{consumo consciente} é, primeiramente, uma responsabilidade do indivíduo consigo mesmo;
        \end{itemize}
      \end{block}
    }

    \frame{\frametitle{Bibliografia}
      \printbibliography[heading=none]
    }

\end{document}

\end{document}

