\section{Introdução}

Entre os anos de 2016 e 2022, a parcela da população brasileira que utiliza a Internet no cotidiano cresceu de 66,1\%
para 87,2\% \cite{carmennery2024}. O crescimento na adoção de \gls{ict}s observado na realidade do nosso país não é um
fenômeno isolado, e sim um reflexo do ritmo acelerado de modernização e digitalização da civilização contemporânea em escala
global \cite{jeffburt2010}. 

Por conseguinte, discussões sobre as possíveis consequências sociais, sistêmicas e
individuais do progresso tecnológico desabalado são um tópico comum nos campos de estudo da economia, sociologia,
geopolítica e estudos ambientais desde meados do século \RNum{20} \cite{kaczynski1995,marcusekellner2001}. Este artigo toma como base a
extensa literatura desenvolvida sobre estes tópicos no último século para articular um processo de análise e reflexão
em relação o impacto da modernização da sociedade como sistema sobre o indivíduo, nos aspectos de vida cotidiana, bem
estar físico e psicológico, autonomia, liberdade e socialização.
