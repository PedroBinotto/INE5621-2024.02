\documentclass[12pt]{article}
\usepackage[portuguese, english]{babel}
\usepackage[T1]{fontenc}
\usepackage[a4paper, left=3cm, right=2cm, top=3cm, bottom=2cm]{geometry}
\usepackage{authblk}
\usepackage{cite}
\usepackage{fontspec}
\usepackage[toc]{glossaries}
\usepackage{graphicx}
\usepackage{hyperref}
\usepackage{setspace}
\usepackage{subcaption}
\usepackage{verbatimbox}
\usepackage[dvipsnames]{xcolor}
\newcommand\myshade{85}
\newcommand{\pprime}{\ensuremath{^{\prime}}}
\addto\captionsportuguese{\renewcommand*\contentsname{Sumário}}
\colorlet{mylinkcolor}{violet}
\colorlet{mycitecolor}{YellowOrange}
\colorlet{myurlcolor}{Aquamarine}

\hypersetup{
    pdftitle={Trabalho Remoto e e-Working - Uma Análise dos Efeitos na Rotina Pessoal, Saúde Mental e Bem-Estar do Trabalhador},
    pdfauthor={Pedro Santi Binotto},
    pdfsubject={Este artigo busca explorar os impactos psicológicos, comportamentais e sociais da adoção do trabalho
    remoto em larga escala e seus efeitos a longo prazo.},
    linkcolor  = mylinkcolor!\myshade!black,
    citecolor  = mycitecolor!\myshade!black,
    urlcolor   = myurlcolor!\myshade!black,
    colorlinks = true,
    filecolor=black,        % color of file links
    linktoc=page            % only page is linked
}%

\newglossaryentry{ict} {
  name={ICT},
  description={
    Tecnologia da Informação e Comunicação (\textbf{ICT}, do inglês: \textit{Information and Communications Technology},
    ou \textbf{TIC}, em português) é um termo extensional para tecnologia da informação (\textbf{TI}) que enfatiza o papel das comunicações
    unificadas e a integração de telecomunicações (linhas telefônicas e sinais sem fio) e computadores, bem como software
    empresarial necessário, middleware, armazenamento e audiovisual, que permitem aos usuários acessar, armazenar, transmitir,
    entender e manipular informações \cite{techTarget2023}.
  }
}

\addto\captionsportuguese{\renewcommand*\contentsname{Sumário}}
\onehalfspacing

\selectlanguage{portuguese} 
\title{%
   Trabalho Remoto e e-Working\\
   \large Uma Análise dos Efeitos na Rotina Pessoal, Saúde Mental e Bem-Estar do Trabalhador }
\author[1]{Pedro Santi Binotto [20200634]\thanks{\texttt{pedro.binotto@grad.ufsc.br}}}
\date{\today}
\affil[1]{Departamento de Informática e Estatística, Universidade Federal de Santa Catarina}

\makeglossaries

\begin{document}
\selectlanguage{portuguese} 
\begin{titlepage}
\maketitle
\thispagestyle{empty}

\begin{abstract} \noindent
  O trabalho das modalidades remota e híbrida tem visto crescimento considerável nos últimos anos, 
  especialmente acelerado em consequência das medidas tomadas por empregadores e instituições governamentais em resposta
  à pandemia de COVID-19 no início desta década. Segundo análises de mercado recentes, para setores da economia do
  conhecimento como TI e finanças, vagas completamente remotas representam a maioria absoluta
  do mercado de trabalho em países desenvolvidos (\textit{McKinsey \& Co.}, 2022). Em vista da mudança radical na % TODO : try to inline citation 
  relação das pessoas com o trabalho causada por essa nova dinâmica, este artigo busca explorar os impactos psicológicos,
  comportamentais e sociais da adoção do trabalho remoto em larga escala e seus efeitos a longo prazo.
\end{abstract}

\end{titlepage}

\newpage
\tableofcontents

\newpage
\printglossary[title=Glossário, toctitle=Glossário]

\newpage
\section{Introdução}

As modalidades de teletrabalho, caracterizadas pela flexibilização do local de trabalho \cite{ilo2020} de forma a
permitir que o empregado realize suas atividades de uma locação remota, externa aos escritórios e complexos de produção
corporativos, tem sido uma realidade desde a viabilização e popularização das \gls{ict}s  \cite{diMartino1990} no
decorrer dos últimos trinta anos \cite{jit2012}, e desde a pandemia de COVID-19, tornou-se comum para uma grande parcela
da força de trabalho de diversas regiões do planeta \cite{apolloTechnical2024}. Este fenômeno tem sido
extensivamente analisado sob perspectivas do empregador \cite{hofschulteBeck2022,sustainability2022} assim como do
empregado \cite{ijerph2021,ntwe2017} de maneira que o resultado destes estudos apresenta uma percepção variada dos
efeitos positivos assim como negativos por ambas as partes.

É no contexto destes estudos que este artigo propõe uma visão geral sintetizada dos efeitos psicológicos cotidianos percebidos
principalmente do referencial do empregado remoto ou híbrido, focando na sustentabilidade e viabilidade dos hábitos
desenvolvidos e analisando os impactos identificados nos padrões de comportamento observados no dia-a-dia do
trabalhador.

\section{Desenvolvimento}
\subsection{Teoria e evidências}

O embasamento teórico para este artigo foi compilado de múltiplos estudos tratando dos sustentabilidade e bem-estar no 
contexto de trabalho remoto conduzidos nos últimos dez anos \cite{mckinsey2022,ijerph2021,ntwe2017,forbes2023},
levando em conta análises da situação pré pandêmica \cite{ntwe2017} assim como estudos conduzidos durante e após a
adoção em massa do teletrabalho durante a pandemia\cite{mckinsey2022,ijerph2021,sustainability2022}. 

\subsection{Adoção do trabalho remoto}
\subsubsection{Pré-pandemia}

Os resultados iniciais dos estudos de ambos os períodos indicam as mesmas tendências à curto prazo: as facilidades associadas à
``não-necessidade'' do deslocamento e a flexibilização do espaço de trabalho são refletidas claramente de maneira positiva,
sendo observável em índices elevados de produtividade e compromisso com o trabalho \cite{ntwe2017} por parte dos
empregados. É importante notar que, de modo geral, o trabalho remoto foi historicamente tratado como um privilégio ou ``luxo'' de
determinado empregado antes do \textit{boom} mundial da pandemia de coronavírus em março de 2020 
\cite{ntwe2017,sustainability2022,diMartino1990}. Inversamente, estudos conduzidos após este período refletem a mudança
de contexto trazida pela pandemia e reconhecem a normalização desta modalidade de trabalho
\cite{hofschulteBeck2022,sustainability2022}.

\subsubsection{Adoção do trabalho remoto durante a pandemia de COVID-19}

No decorrer da pandemia, observamos o trabalho remoto na forma de \textit{home office} tornar-se um ``novo
normal'' compulsório para a uma parcela considerável da população economicamente ativa por um período mais longo do que
anteriormente observado \cite{mckinsey2022,sustainability2022,phillips2020}. Assim, ficaram claros alguns dos efeitos adversos da forma com
que o trabalho remoto tem sido conduzido desde então: em muitos casos, o trabalho exclusivamente remoto durante
períodos estendidos está correlacionado à níveis de estresse elevados, redução do equilíbrio entre vida pessoal e profissional,
assim como redução da percepção de satisfação profissional \cite{ijerph2021,forbes2023}. Apesar disso, nota-se que o trabalho
remoto mantém a correlação com níveis mais altos de engajamento e rendimento profissional \cite{ijerph2021,ntwe2017}.
Não obstante, empregadores ainda reportam desafios ao se adaptar à nova realidade dessa espécie de trabalho, adotando
novas medidas de monitoramento motivadas pela percepção de uma linha de comunicação enfraquecida e/ou falta de observabilidade
sobre os empregados \cite{hofschulteBeck2022}.

\subsubsection{Impacto psicológico do trabalho remoto por períodos estendidos}



% - Pontuar isolamento social, switch-off, etc...
% - Definir persona do trabalhador remoto
% - Relacionar com profissao de TI, outsourcing/offshoring
% - perspectivas futuras

\section{Conclusão}
\paragraph{}

Conteúdo de parágrafo

\newpage
\section{Referências Bibliográficas}
\bibliographystyle{plain}
\bibliography{references}

\end{document}


